\chapter*{Abstract}

The field of Neural Radiance Fields (NeRF) has experienced a surge in research and development over the past years, with significant enhancements to model performance and an increasing scope of application areas. Amidst this growth, the use of NeRFs in autonomous driving systems has emerged as a promising area of exploration, as NeRFs can enable the generation of photorealistic edge case scenarios and an environment to evaluate autonomous vehicles.

The primary research goal of this thesis is to design and develop an end-to-end pipeline for generating NeRFs, leveraging vehicle-captured video sequences and corresponding camera poses with varying degrees of accuracy.

Initially, a data capture pipeline is created for CARLA, providing synthetic data from a controlled environment. Connecting this data capture pipeline with a NeRF pipeline facilitates the creation of a performance baseline for further experiments. Having obtained a baseline and an end-to-end pipeline, the thesis explores large-scale NeRF approaches and implements a performant prototype. Finally, the pipeline is extended to enable the input of real data captured by a specialized vehicle with accurate GNSS and high-resolution cameras.

Most of the findings are consistent across synthetic and real data; the configuration of the data-capture significantly affects both the data and the resulting NeRF's quality; a large-scale approach where a scene is learned by multiple smaller NeRFs, contrary to a single NeRF, performs better; joint camera pose optimization efficiently reduces the impact of imperfect camera poses, but approximating the poses with SfM-tools a priori demonstrates superior results.

Investigating the application of rendering novel views for edge case scenarios we find that the renderings do not match the original images' quality. However, they are largely successful in producing clear and structurally accurate renderings, reaffirming NeRFs potential and the effectiveness of the approach.