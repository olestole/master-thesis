\chapter{Introduction}

\begin{comment}    
1.1 Motivation

1.2 Goal and research questions
The overall goal of this research is ...
RQ1, RQ2, ...
- Pipeline
- Eksperimenter

1.3 Thesis outline
\end{comment}

% and modify an open-source large-scale NeRF pipeline to create a virtual environment of parts of Trondheim.
%The advance in Neural Radiance Fields (NeRFs) proposes a new method for reconstructing 3D scenes from 2D images.
%By leveraging NeRFs we can create a pipeline that efficiently reconstructs 3D scenes. The reconstructed scenes have many applications, and could be beneficial in applications that require simulated virtual worlds.

\begin{comment}
\section{Motivation}
- Surge in NerF research. The models have become exponentially better and more performant recently.
- Enables efficient reconstruction of 3D scenes
- Large-scale Nerf has important applications, e.g. in autonomous driving. It's already used by Wayve in this respect.
- I want to see if I can create a functional pipeline to capture data and reconstruct larger scenes. 
- In order to do so there are a few steps:
    - Create a data capture pipeline in a controllable virtual environment
    - Connect the data capture pipeline to a NeRF-pipeline
    - Construct a baseline that can be used to compare future iterations against.
    - Extend the pipeline to enable input of real data capture
\end{comment}

%By leveraging NeRFs we can create a pipeline that efficiently reconstructs 3D scenes. The reconstructed scenes have many applications, and could be beneficial in applications that require simulated virtual worlds. In order to create a functional pipeline I want to use synthetic data from a controllable environment and later extend the pipeline to enable real data capture.

\section{Motivation}
The field of Neural Radiance Fields (NeRF) has experienced a surge in research and development over the past years, with models becoming exponentially better and more performant \cite{debbagh_neural_2023}. These models have demonstrated impressive, photo-realistic reconstruction and novel view synthesis of 3D scenes, given a set of posed 2D camera images. This surge in research has opened up new frontiers in many applications including the expansion of mapping and navigation capabilities (Google Immersive View\footnote{\url{https://blog.google/products/maps/three-maps-updates-io-2022/}}), computer graphics and visual effects (Volinga\footnote{\url{https://volinga.ai/}}), asset creation (Luma AI\footnote{\url{https://lumalabs.ai/}}), and \acrshort{ad} (Wayve\footnote{\url{https://wayve.ai/neural-rendering/}}). 

Looking further into the realm of \acrshort{ad}, one area of application that could benefit from these advances stands out. The evaluation of \acrshort{ad} systems often involves the re-simulation of scenarios that have been encountered in the past. Yet, the vehicle's path could be altered by even the slightest divergence from the original incident, thus necessitating high-quality rendering of novel views along the altered trajectory \cite{tancik_block-nerf_2022}. Wayve, a pioneer in the autonomous vehicle industry, has already integrated NeRF technology into their system for the purpose of rendering novel views along altered paths. However, as with most commercial applications, their tools and code remain proprietary leaving them inaccessible for wider research purposes.

In order to explore the use of NeRFs in applications such as \acrshort{ad} that could benefit from the generation of high-quality renders of novel views, this thesis seeks to design and develop an end-to-end pipeline capable of capturing data from a scene, and subsequently generating a \acrshort{nerf} representing the same scene. 

The first part of the project focuses on creating a data capture pipeline for a vehicle in a controllable environment. Having created a pipeline for data capture, it is connected to a pipeline for training, rendering, and evaluating NeRFs. The end-to-end pipeline is then used to configure the data capture and NeRF-settings in order to obtain a performance baseline. Further experiments investigating approaches to deal with noise or increasing scene-size is conducted, before the pipeline is extended to enable input of real data.

\begin{comment}
Looking deeper into autonomous driving, one of the many applications where NeRF could prove beneficial
"Let's look further into one of the applications where NeRF could be beneficial, e.g. autonomous driving."
Looking deeper into autonomous driving, an application that could benefit from these advances
Zooming in on an application that could benefit from these advances, autonomous driving fits well.
\end{comment}

% In order to explore the application of NeRFs in autonomous driving this thesis seeks to design and develop an end-to-end pipeline capable of capturing data from a scene, and subsequently generating a Neural Radiance Field (NeRF) representing the same scene. The resulting NeRF can in turn be used to create photo-realistic renders of unseen scenarios and trajectories, expanding the dataset for training autonomous agents and potentially enhancing their capabilities.

%The first part of the project will focus on creating a data capture pipeline for a vehicle in a controllable environment. Having created a pipeline for data capture, it will be connected to a pipeline for training, rendering, and evaluating NeRFs. The end-to-end pipeline will be used to configure the data capture and NeRF-settings in order to obtain a performance baseline. Further experiments investigating approaches to deal with noise or increasing scene-size will be conducted, before the pipeline is extended to enable input of real data.






% CARLA, an open-source simulator designed for autonomous driving research will be utilized for this purpose.
% Nerfstudio, an open-source framework/API that streamlines the creation, training, and visualization of NeRFs will be utilized for this purpose.

\begin{comment}
The field of Neural Radiance Fields (NeRF) has experienced a surge in research and development over the past years, with models becoming exponentially better and more performant. These models have demonstrated impressive capabilities in efficiently reconstructing 3D scenes from 2D images, opening up new frontiers in many applications including autonomous driving. As an example, Wayve, a pioneer in the self-driving vehicle industry, has already integrated NeRF technology into their system (\textbf{Cite Nvidia GTC Talk}). As with most large-scale NeRF implementations with commercial applications, their tools and code remain proprietary, rendering them inaccessible for wider research purposes. However, this advancement underscores the potential significance of the technology and motivates further exploration of its capabilities and potential applications.

One particular area of interest is the ability to create a pipeline that captures data and reconstructs larger scenes, facilitating the application of NeRF models in large-scale environments. A functional pipeline could further enhance the performance of systems like Wayve's and lead to broader implications in other sectors requiring 3D scene reconstruction.

There are multiple key steps to achieving the goal of designing and developing an end-to-end pipeline that enables the capture and reconstruction of large 3D scenes. A pragmatic approach would first create a data capture pipeline in a controllable virtual environment, which in turn is connected to a NeRF pipeline. Having constructed the pipeline it could be used to create a baseline to compare future iterations of scene capture and NeRF-settings, allowing steady improvement. With a functional pipeline, and an acquired baseline, the pipeline could be extended to enable the input of real data.

Given the importance and potential of this research in the evolving landscape of 3D scene reconstruction, this thesis seeks to explore these steps and further contribute to the knowledge and application of NeRF technology.
\end{comment}









\section{Goal and Research Questions}

The primary research goal of this thesis is to design and develop an end-to-end pipeline for generating NeRFs, leveraging vehicle-captured video sequences and corresponding camera poses with varying degrees of accuracy. In order to achieve this goal, four research questions were posed:


\begin{description}[leftmargin=!,labelwidth=\widthof{RQ 1:}]
% CREATING A BASELINE
\item[\textbf{RQ 1:}] What are the critical factors that need to be considered when capturing data for training NeRF models, and how do they impact the performance of the resulting models?

% CAMERA POSE OPTIMIZATION AND NOISY GPS/GNSS READINGS
\item[\textbf{RQ 2:}] How does the initial camera pose accuracy and segment length impact the final reconstruction? Can rough initial camera poses be optimized to achieve comparable results to those obtained from tools such as COLMAP?

% DIFFERENT MODELS
\item[\textbf{RQ 3:}] How do different NeRF methods (Instant-ngp\cite{muller_instant_2022}, Mip-NeRF\cite{barron_mip-nerf_2021}, Nerfacto\cite{tancik_nerfstudio_2023}) perform on unbounded scenes in terms of reconstruction quality and computational efficiency?

% IMPLEMENTATION OF BLOCK NERF
\item[\textbf{RQ 4:}] What are the technical challenges and considerations for implementing a functional approach for large-scale NeRF within the Nerfstudio API, and how does it compare to approaches not optimized for large-scale in terms of scalability, efficiency, and rendering quality?

% EXTENDING TO REAL DATA

\end{description}

\begin{comment}
The research goal for this thesis is:
\begin{description}[leftmargin=!,labelwidth=\widthof{RQ:}]
\item[\textbf{RG:}] Design and develop an end-to-end pipeline for generating NeRFs, leveraging vehicle-captured video sequences and corresponding camera poses with varying degrees of accuracy.
\end{description}

In order to achieve this goal, four research questions were posed:
\end{comment}




\section{Research Method}
The chosen research method for this report is experimental research. The experiments in this report will compare different methods, techniques, and configurations used to capture and process data, and subsequently used to train NeRFs. Observation will be used for the quantitative and qualitative evaluation of the results. The quantitative evaluation will span common image reconstruction quality metrics, whereas the qualitative assessment primarily will consist of analyzing video rendering outputs looking for abnormalities or other interesting findings.







\section{Contributions}
One of the primary contributions of this master's thesis is the establishment of a pipeline from CARLA \cite{Dosovitskiy17} to Nerfstudio \cite{nerfstudio}. This pipeline serves as a tool for conducting a variety of experiments, which could further be applied to real-world scenarios. By permitting the manipulation of various experiment settings, such as the camera setup, vehicle speed, route, and image resolution, this pipeline streamlines the process. Furthermore, the pipeline's data output follows standard conventions, thereby enabling the training of NeRFs on synthetic data captured in CARLA. This consequently facilitates the refinement of settings to enhance the resulting image synthesis based on the evaluation of the NeRF. Furthermore, the pipeline is also extended to allow the input of real data captured by the \acrfull{naplab} car \cite{naplab}.

In addition to the pipeline, this thesis introduces a baseline for NeRFs trained on synthetic data captured in CARLA. This baseline can be used to augment both the data capture and the NeRF models on synthetic data. It offers a platform for experimenting with data capture and NeRF settings, and its evaluation metrics (\acrshort{psnr}, \acrshort{ssim}, and \acrshort{lpips}) align with those widely used in NeRF research, ensuring comparability.

Another contribution lies in the development of a \acrfull{poc} for large-scale NeRF approach with the Nerfstudio API. By creating a naive Block-NeRF implementation, it demonstrates how such an approach can dramatically enhance the quality of large-scale NeRFs. This \acrshort{poc} provides a testing ground for determining crucial parameters when capturing large-scale data for NeRFs, such as the segment size, the overlap between the blocks, and image merging techniques.

Lastly, the thesis includes the creation of a utility for generating side-by-side view using FFMPEG\cite{tomar2006converting}. This enables the creation of a comparative view of the rendered NeRF and the ground truth. This addition not only facilitates the qualitative assessment of the resulting NeRF but also enhances the overall analysis process.

\begin{comment}
\begin{itemize}
    \item Pipeline from CARLA to Nerfstudio
    \begin{itemize}
        \item Enables experimenting with data capture techniques that can later be applied to real world scenarios.
        \item Enables running multiple experiments with different experiment settings, e.g. camera setup, vehicle speed, route, image resolution, etc., in a streamlined way.
        \item The data output from the experiment-pipeline in CARLA is in a format supported by most NeRFs. Enables training NeRFs on the synthetic data captured in CARLA, and, based on the evaluation of the NeRF, tweak settings to improve the resulting image synthesis.
    \end{itemize}
    \item A baseline for NeRFs trained on synthetic data captured in CARLA. 
    \begin{itemize}
        \item Can be used to further improve both the data capture and the NeRF-models on synthetic data.
        \item Can be used to experiment with data capture- and NeRF-settings.
        \item The metrics used to evaluate the baseline (PSNR, SSIM and LPIPS) are widely used throughout NeRF-research and makes comparable.
    \end{itemize}
    \item Block-NeRF in Nerfstudio API
    \begin{itemize}
        \item Creates a naive Block-NeRF implementation with the Nerfstudio API, demonstrating how such an approach can substantially increase the quality of large scene NeRFs.
        \item The PoC allows testing which parameters are important when capturing large-scale data for NeRFs. E.g. the segment size, overlap between the blocks, and image merging techniques.
    \end{itemize}
    \item Side-by-side view
    \begin{itemize}
        \item Leverage FFMPEG to create a script for generating side-by-side views of the rendered NeRF and the ground truth.
        \item Makes qualitative assessment of the resulting NeRF easier.
    \end{itemize}
\end{itemize}

\end{comment}



\section{Report Outline}

\begin{description}[leftmargin=!,labelwidth=\widthof{Chapter 1:}]
\item[\textbf{Chapter 1 - Introduction:}]
Presents the study's motivation, goal, research method, contributions, and research questions.

\item[\textbf{Chapter 2 - Background and Related Work:}]
Provides an overview of preliminary methods, techniques, tools, frameworks, and metrics in order to establish a common ground for successive chapters.

\item[\textbf{Chapter 3 - Methods:}]
Provides details about the multi-stage process necessary for the designed and developed end-to-end pipeline. Sections span in-depth discussion and overview of the virtual environment, the data capture pipeline, the NeRF pipeline, the extension to large-scale scenes, and lastly the extension from synthetic to real data.

\item[\textbf{Chapter 4 - Experiments and Results:}]
Presents the conducted experiments and subsequent results.

\item[\textbf{Chapter 5 - Discussion:}]
Presents a detailed discussion of the results and their implications in the context of the research questions.

\item[\textbf{Chapter 6 - Conclusion and Future Work:}]
Summarizes the main findings of the study and further suggests a direction for future work.
\end{description}




% Comments

% Research questions
\begin{comment}
1. What other experiments could have been included in defining the baseline for capturing data for NeRFs?
2. How could the process of choosing which experiments to include in the baseline have been done differently?
3. What factors affect the quality of the data captured by the mounted cameras in the context of training NeRFs?
4. How does the speed of the vehicle affect the quality of the data captured by the mounted cameras in the context of training NeRFs?
5. How effective is the camera pose optimization step in the Nerfacto-pipeline in reducing noise in the rendered images?
6. How does the use of Gaussian noise in the experiments affect the evaluation of the impact of camera optimization in the Nerfacto-pipeline?
7. How does the Block-NeRF implementation compare to the traditional NeRF implementation in terms of efficacy and computational requirements?
8. What improvements can be made to the Block-NeRF implementation to further mitigate the differences in quality between blocks?
9. How was the implementation of the CARLA to NerfStudio pipeline executed?


1. "How can imperfections in the dataset affect the performance of a CARLA to Nerfstudio pipeline leveraging Neural Radiance Fields and how can these be mitigated?"
2. "What is the impact of various factors (like vehicle speed, number of frames, image size, etc.) on the effectiveness of the CARLA to Nerfstudio pipeline leveraging Neural Radiance Fields?"
3. "What are the improvements gained by extending the CARLA-baseline to support large-scale scenes with Block-NeRF in the context of a Neural Radiance Fields pipeline?"


1. How can the pipeline from CARLA to Nerfstudio be further optimized to enhance the realism and accuracy of data capture in simulated environments?
2. How does the variability in experiment settings (e.g., camera setup, vehicle speed, route, image resolution) influence the quality of data output and the subsequent NeRF modeling in the CARLA-Nerfstudio pipeline?
3. How can we effectively measure and quantify the improvement in image synthesis resulting from tweaks to settings based on NeRF evaluation?
4. What role does the established baseline play in advancing NeRF-models on synthetic data captured in CARLA, and how can it be utilized more efficiently?
5. How can the data capture settings be optimized to improve the quality and consistency of synthetic data for NeRF training, leveraging the baseline established?
6. Which metrics, in addition to PSNR, SSIM, and LPIPS, can be employed to provide a more comprehensive evaluation of the NeRF baseline?
7. How does the integration of Block-NeRF in the Nerfstudio API improve the quality of large scene NeRFs, and what are the key parameters influencing this improvement?
8. What strategies can be used to identify and modify the essential parameters (such as segment size, block overlap, image merging techniques) when capturing large scale data for NeRFs using Block-NeRF in the Nerfstudio API?
9. How does the side-by-side view feature leveraging FFMPEG enhance the qualitative assessment of the resulting NeRF, and how can this be further improved?
10. What potential future extensions can be envisioned for the CARLA-Nerfstudio pipeline to accommodate real data, moving beyond synthetic data environments, and what would be the challenges associated with such extensions?
\end{comment}

\begin{comment}
Gammelt forslag:
How does the accuracy of initial camera poses and segment size influence the effectiveness of camera pose optimization, and is it possible to achieve equivalent results by optimizing rough camera poses directly, as opposed to pre-processing an approximation of accurate camera poses with tools such as COLMAP?

Gabriel:
What is the influence of the initial camera pose and segment length on the final reconstruction? Does it help to have a rough estimate of the camera pose instead of using tools such as COLMAP on image sequences without camera position information?

ChatGPT:
How does the precision of initial camera poses and the length of image sequences impact the outcome of camera pose optimization? Is it viable to optimize rough camera pose estimations directly and still achieve results comparable to those obtained when applying tools like COLMAP for preprocessing and estimating accurate camera poses?
\end{comment}

% Research method
\begin{comment}
The research method for this study primarily involves experimental design. We will conduct a series of experiments to compare the effectiveness of different data capturing and processing techniques, as well as various configurations for training NeRF models. Our experimental procedure consists of two major components: a quantitative evaluation and a qualitative analysis.

In the quantitative evaluation, we will use recognized image reconstruction quality metrics such as the Peak Signal-to-Noise Ratio (PSNR) and the Structural Similarity Index Measure (SSIM) to assess the performance of the models.

For the qualitative assessment, we will conduct an in-depth analysis of the video rendering outputs generated by the models. We aim to identify any abnormalities or particularly noteworthy findings through this detailed visual inspection. The results of both quantitative and qualitative evaluations will be used to draw conclusions about the efficiency, scalability, and rendering quality of different models and configurations.
\end{comment}


% Contributions
\begin{comment}
- Pipeline from CARLA to Nerfstudio
    - Enables experimenting with data capture techniques that can later be applied to real world scenarios.
    - Enables running multiple experiments with different experiment settings, e.g. camera setup, vehicle speed, route, image resolution, etc., in a streamlined way.
    - The data output from the experiment-pipeline in CARLA is in a format supported by most NeRFs. Enables training NeRFs on the synthetic data captured in CARLA, and, based on the evaluation of the NeRF, tweak settings to improve the resulting image synthesis.
- A baseline for NeRFs trained on synthetic data captured in CARLA. 
    - Can be used to further improve both the data capture and the NeRF-models on synthetic data.
    - Can be used to experiment with data capture- and NeRF-settings.
    - The metrics used to evaluate the baseline (PSNR, SSIM and LPIPS) are widely used throughout NeRF-research and makes comparable.
- Block-NeRF in Nerfstudio API
    - Creates a naive Block-NeRF implementation in the Nerfstudio API, demonstrating how such an approach can substantially increase the quality of large scene NeRFs.
    - The PoC allows testing which parameters are important when capturing large scale data for NeRFs. E.g. the segment size, overlap between the blocks, image merging techniques.
- Side-by-side view
    - Leverage FFMPEG to create a script for generating side-by-side views of the rendered NeRF and the ground truth.
    - Makes qualitative assessment of the resulting NeRF easier.
\end{comment}